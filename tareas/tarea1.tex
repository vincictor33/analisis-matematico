\documentclass[12pt]{article}

\addtolength{\hoffset}{-2.25cm}
\addtolength{\textwidth}{4.5cm}
\addtolength{\voffset}{-2.5cm}
\addtolength{\textheight}{5cm}
\setlength{\parskip}{0pt}
\setlength{\parindent}{15pt}

\usepackage{amsthm}
\usepackage{amsmath}
\usepackage{amssymb}
\usepackage{enumitem}
\usepackage[colorlinks = true, linkcolor = blue, citecolor = blue, final]{hyperref}

\usepackage{graphicx}
\usepackage{multicol}
\usepackage{ marvosym }
\usepackage{wasysym}
\usepackage{tikz}
\usepackage{xcolor} 
\usepackage{CJKutf8}
\usepackage{tabularx}
\usepackage{float}
\usepackage{pgfplots}
\usepackage{fancyhdr}
\usepackage{amsfonts}
\usepackage{physics}
\usepackage{amsmath}
\usetikzlibrary{patterns}

\newcommand{\subscript}[2]{$#1 _ #2$}
\newcommand{\ds}{\displaystyle}
\newcommand\N{\ensuremath{\mathbb{N}}}
\newcommand\R{\ensuremath{\mathbb{R}}}
\newcommand\Z{\ensuremath{\mathbb{Z}}}
\renewcommand\O{\ensuremath{\emptyset}}
\newcommand\Q{\ensuremath{\mathbb{Q}}}
\newcommand\C{\ensuremath{\mathbb{C}}}
\DeclareMathOperator{\spn}{span}

\pgfplotsset{compat=1.18} 
\setlength{\parindent}{0in}

\pagestyle{empty}

\begin{document}

\thispagestyle{empty}

{\scshape Análisis $\R$eal} \hfill {\scshape \large Tarea I} \hfill {\scshape Victor Ortega}
 
\smallskip

\hrule

\bigskip

\bigskip

\theoremstyle{definition}
\newtheorem*{definition}{Definición}

\theoremstyle{definition}
\newtheorem*{dem}{Demostración}

\theoremstyle{definition}
\newtheorem*{notation}{Notación}

\theoremstyle{definition}
\newtheorem*{theorem}{Teorema}

\theoremstyle{definition}
\newtheorem*{lema}{Lema}

\theoremstyle{remark}
\newtheorem*{observation}{Observación}

\theoremstyle{remark}
\newtheorem*{example}{Ejemplo}


\begin{enumerate}[label=\textbf{\arabic*}.]

\item Sea $\rho$ una métrica en un conjunto $X$. Demostrar que las siguientes funciones también son métricas:

\begin{enumerate}[label=\alph*)]
    \item ${\rho}_{1}(x,y)=\frac{\rho(x,y)}{1+\rho(x,y)}$
    \item ${\rho}_{3}(x,y)=\min(\{1,\rho(x,y))\})$
\end{enumerate}

\begin{proof}
    

\begin{enumerate}[label=(\subscript{D}{{\arabic*}})]
\item  Sean $x,y \in \mathbb{R}^{n}$ P.D. ${\rho}_{1}(x,y) \geqslant 0$. 
        
Notemos que, ${\rho}(x,y) \geqslant 0$

\begin{equation*}
     \Rightarrow 1+{\rho}(x,y) \geqslant 0
\end{equation*}


Si $a\geqslant0$ y $b\geqslant0$ $\Rightarrow \frac{a}{b} \geqslant 0$

\begin{equation*}
    \Rightarrow \frac{{\rho}(x,y)}{1+{\rho}(x,y)} \geqslant 0    \Rightarrow {\rho}_{1}(x,y) \geqslant 0
\end{equation*}

\item  Sean $x,y \in \mathbb{R}^{n}$ P.D. ${\rho}_{1}(x,y)=0 \Leftrightarrow x=y$

\begin{equation*}
    {\rho}_{1}(x,y)=0 \Leftrightarrow \frac{{\rho}(x,y)}{1+{\rho}(x,y)} = 0
\end{equation*}

    Si $b \neq 0$ $\Rightarrow \frac{a}{b}=0 \Leftrightarrow a = 0$

\begin{equation*}
    \frac{{\rho}(x,y)}{1+{\rho}(x,y)} = 0 \Leftrightarrow {\rho}(x,y) = 0 \Leftrightarrow x = y
\end{equation*}

\item  Sean $x,y \in \mathbb{R}^{n}$ P.D. ${\rho}_{1}(x,y)={\rho}_{1}(y,x)$

\begin{equation*}
    {\rho}_{1}(x,y)= \frac{{\rho}(x,y)}{1+{\rho}(x,y)}=\frac{{\rho}(y,x)}{1+{\rho}(y,x)}= {\rho}_{1}(y,x)
\end{equation*}

\item  Sean $x,y \in \mathbb{R}^{n}$ P.D. ${\rho}_{1}(x,z) \leqslant {\rho}_{1}(x,y) +{\rho}_{1}(y,z)$

\begin{equation*}
    {\rho}_{1}(x,z) =\frac{{\rho}(x,z)}{1+{\rho}(x,z)} = \frac{1+{\rho}(x,z)-1}{1+{\rho}(x,z)}
\end{equation*}

\begin{equation*}
    = \frac{1+{\rho}(x,z)}{1+{\rho}(x,z)}-\frac{1}{1+{\rho}(x,z)}=1-\frac{1}{1+{\rho}(x,z)}
\end{equation*}

\begin{equation*}
    \leqslant 1-\frac{1}{1+{\rho}(x,y)+{\rho}(y,z)}=\frac{1+{\rho}(x,y)+{\rho}(y,z)-1}{1+{\rho}(x,y)+{\rho}(y,z)}
\end{equation*}

\begin{equation*}
    =\frac{{\rho}(x,y)+{\rho}(y,z)}{1+{\rho}(x,y)+{\rho}(y,z)}=\frac{{\rho}(x,y)}{1+{\rho}(x,y)+{\rho}(y,z)}+\frac{{\rho}(y,z)}{1+{\rho}(x,y)+{\rho}(y,z)}
\end{equation*}

\begin{equation*}
    \leqslant \frac{{\rho}(x,y)}{1+{\rho}(x,y)}+\frac{{\rho}(y,z)}{1+{\rho}(y,z)}={\rho}_{1}(x,y)+{\rho}_{1}(y,z) 
\end{equation*}

\begin{equation*}
    \Rightarrow {\rho}_{1}(x,z) \leqslant {\rho}_{1}(x,y)+{\rho}_{1}(y,z)
\end{equation*}
\end{enumerate}

$\therefore {\rho}_{1}(x,y)$ es una métrica en $\mathbb{R}^{n}$.
\end{proof}

\begin{proof}
\begin{enumerate}[label=(\subscript{D}{{\arabic*}})]
\item  Sean $x,y \in \mathbb{R}^{n}$ P.D. ${\rho}_{3}(x,y) \geqslant 0$

Si $\rho(x,y)>1 \Rightarrow {\rho}_{3}(x,y)=1$. Si  $\rho(x,y))\leqslant 1 \Rightarrow {\rho}_{3}(x,y)=\rho(x,y)$. Tenemos los siguientes casos.

\begin{enumerate}[label=\alph*)]
    
\item  $\rho(x,y)) > 1 \Rightarrow {\rho}_{3}(x,y)=1 $

\begin{equation*}
    1 > 0 \Rightarrow {\rho}_{3}(x,y) \geqslant 0
\end{equation*}

\item $\rho(x,y)) \leqslant 1 \Rightarrow {\rho}_{3}(x,y)=\rho(x,y)) $ Por definición,  $\rho(x,y) \geqslant 0$, $\Rightarrow {\rho}_{3}(x,y)\geqslant 0$

\end{enumerate}

\item Sean $x,y \in \mathbb{R}^{n}$ P.D. ${\rho}_{3}(x,y) = 0 \Leftrightarrow x=y$

 \begin{equation*}
     {\rho}_{3}(x,y) = 0 \Leftrightarrow \min(\{1,\rho(x,y)\}) = 0 \Leftrightarrow \rho(x,y) = 0 \Leftrightarrow x=y
 \end{equation*}

\item   Sean $x,y \in \R^{n}$ P.D. ${\rho}_{3}(x,y) ={\rho}_{3}(y,x) $

\begin{equation*}
     {\rho}_{3}(x,y) = \min(\{1,\rho(x,y)\}) = \min(\{1,\rho(y,x)\}) = {\rho}_{3}(y,x)
\end{equation*}

\item  Sean $x,y \in \mathbb{R}^{n}$ P.D. ${\rho}_{3}(x,z) \leqslant {\rho}_{3}(x,y) +{\rho}_{3}(y,z)$. Tenemos los siguientes casos.

\begin{enumerate}[label=\alph*)]
    \item  Si $\rho(x,y)) > 1 \Rightarrow {\rho}_{3}(x,y)=1 $ o ${\rho}(y,z) > 1 \Rightarrow {\rho}_{3}(y,z)=1 $

Notemos que, por definición, ${\rho}_{3}(x,z) \leqslant 1$

\begin{equation*}
    {\rho}_{3}(x,y) +{\rho}_{3}(y,z) \geqslant 1
\end{equation*}
\begin{equation*}
    {\rho}_{3}(x,z) \leqslant 1 \leqslant {\rho}_{3}(x,y) +{\rho}_{3}(y,z) \Rightarrow {\rho}_{3}(x,z) \leqslant {\rho}_{3}(x,y) +{\rho}_{3}(y,z)
\end{equation*}

\smallskip

\item Si $\rho(x,y)) < 1 \Rightarrow {\rho}_{3}(x,y)=\rho(x,y)) $ o ${\rho}(y,z) < 1 \Rightarrow {\rho}_{3}(y,z)={\rho}(y,z) $

\begin{equation*}
    {\rho}_{3}(x,y)+{\rho}_{3}(y,z)= \rho(x,y))+{\rho}(y,z) \geqslant {\rho}(x,z) \geqslant {\rho}_{3}(x,z) \Rightarrow
\end{equation*}
\begin{equation*}
    {\rho}_{3}(x,z) \leqslant {\rho}_{3}(x,y) +{\rho}_{3}(y,z)
\end{equation*}

\end{enumerate}
\end{enumerate}

$\therefore {\rho}_{3}(x,y)$ es una métrica en $\mathbb{R}^{n}$
    
\end{proof}

\item \begin{enumerate}[label=\alph*)]
    \item Sea $\rho : X \times X \to \R$ una métrica. Prueba la desigualdad tetrahédrica

    \begin{equation*}
        \abs{\rho(x,y)-\rho(z,w)} \leqslant \rho(x,z) +\rho(y,w)
    \end{equation*}

    $\forall \: x,y,z,w \in X$
    \item Sea $\rho : X \times X \to \R$ una función $\backepsilon$
    \begin{enumerate}
        \item $\forall \: x \in X \Rightarrow \rho(x,x) = 0  $
        \item $\forall \: x \neq y \in X \Rightarrow \rho(x,y) = \rho(y,x) > 0  $
        \item Satisface la desigualdad tetrahédrica
    \end{enumerate}

    Pruebe que $\rho$ es una métrica
    \end{enumerate}

\begin{proof}
    Sea $x,y,z,w \in X$ arbitrario. Como $\rho$ es una métrica en $X$ s.t.q.

    \begin{equation*}
        \rho(x,y) \leqslant \rho(x,w) + \rho(w,y) \leqslant (\rho(x,z) + \rho(z,w)) + \rho(y,w) 
    \end{equation*}
    \begin{equation*}
        \Rightarrow \rho(x,y) - \rho(w,z) \leqslant  \rho(x,z)  + \rho(y,w)
    \end{equation*}

    Además

    \begin{equation*}
        \rho(z,w) \leqslant \rho(z,y) + \rho(y,w) \leqslant (\rho(z,x) + \rho(x,y)) + \rho(w,y)
    \end{equation*}
    \begin{equation*}
        \Rightarrow \rho(w,z) - \rho(x,y) \leqslant  \rho(x,z)  + \rho(y,w)
    \end{equation*}

    Tanto $\rho(x,y)-\rho(z,w)$ como $-(\rho(x,y)-\rho(z,w)) \leqslant \rho(x,z) +\rho(y,w) \Rightarrow$ con la función $\abs{\cdot}$ s.t.q.

    \begin{equation*}
        \abs{\rho(x,y)-\rho(z,w)} \leqslant \rho(x,z) +\rho(y,w)
    \end{equation*}
\end{proof}

\begin{proof}
\begin{enumerate}[label=(\subscript{D}{{\arabic*}})]
\item Por definición de $\rho$ s.t.q $\forall \: x, y \in X \Rightarrow \rho(x,y) \geq 0 $
\item Si $x=y$, por el inciso i. $\rho(x,x) = 0 \Rightarrow$ necesariamente $x=y$, porque de lo contrario, por ii. $\rho(x,y) > 0$
\item Se sigue de i. y ii.
\item Sean $x,y,z \in X$ elementos arbitrarios
\begin{equation*}
    \rho(x,y) = \abs{\rho(x,y)} = \abs{\rho(x,y)+\rho(x,z)-\rho(x,z)}
\end{equation*}
\begin{equation*}
    \leqslant  \abs{\rho(x,z)} +  \abs{\rho(x,y)-\rho(x,z)}
\end{equation*}
Por la desigudaldad tetrahédrica s.t.q.
\begin{equation*}
    \abs{\rho(x,y)-\rho(x,z)} \leqslant \rho(x,x) +\rho(y,z)
\end{equation*}
\begin{equation*}
    \Rightarrow \rho(x,y) \leqslant \rho(x,z) + \rho(x,x) +\rho(y,z) = \rho(x,z) +\rho(y,z)
\end{equation*}
\end{enumerate}
$\therefore \rho$ es una métrica en $X$
\end{proof}

\item ¿Qué condiciones debe satisfacer una función continua $f : \R \to \R$ definida sobre $\R$ para que en la recta real se pueda dar una métrica por medio de la igualdad $\rho(x, y) = \abs{f(x) - f(y)}?$

Esta función debe de ser injectiva, es decir $f(x) = f(y) \Leftrightarrow x = y$

\begin{proof}
\begin{enumerate}[label=(\subscript{D}{{\arabic*}})]
\item Sea $x,y \in \R \Rightarrow \abs{f(x) - f(y)} \geqslant 0$ por propiedades de $\abs{\cdot}$
\item Sea $x,y \in \R$ Como $f$ es inyectiva, s.t.q.
\begin{equation*}
    \rho(x, y) = 0 \Leftrightarrow \abs{f(x) - f(y)} \Leftrightarrow f(x) = f(y) \Leftrightarrow x = y 
\end{equation*}
\item Sea $x,y \in \R$. Por propiedades de $\abs{\cdot}$ s.t.q.
\begin{equation*}
     \rho(x, y) = \abs{f(x) - f(y)} = \abs{f(y) - f(x)} = \rho(y,x) 
\end{equation*}
\item  Sea $x,y,z \in \R$. Por propiedades de $\abs{\cdot}$ s.t.q.
\begin{equation*}
    \rho(x,z) = \abs{f(x)-f(z)} \leqslant \abs{f(x)-f(y)}+\abs{f(y)-f(z)} = \rho(x,y)+\rho(y,z)
\end{equation*}
\end{enumerate}
$\therefore \rho$ es métrica en $\R$
\end{proof}

\item Suponga que ($X,d$) es un espacio métrico y fije una función $f : [0, \infty] \to  [0, \infty]$ estrictamente creciente $\backepsilon f(0) = 0$. Demuestre que si $f$ es una función subaditiva (es decir, si $\forall \: x,y \in [0,\infty]$ se teiene la desigualdad $f(x+y) \leqslant f(x) + f(y) \Rightarrow f \circ d : X \times X \to [0,\infty]$ es una métrica en $X$.

\begin{proof}
\begin{enumerate}[label=(\subscript{D}{{\arabic*}})]
\item Sean $x,y \in X$. Como $d$ es una métrica en $X \Rightarrow d(x,y) \geqslant 0$. Por otra parte, como $f$ es estrictamente creciente $f(t) \geqslant 0 \: \forall \: t\in [0,\infty]$. Por lo cual $(f \circ d) (x,y) = f(d(x,y)) \geqslant 0$
\item Sean $x,y \in X$. Por hipótesis $f(t) = 0 \Leftrightarrow t = 0 \: \forall \: t\in [0,\infty]$. Usando esto s.t.q.
\begin{equation*}
    (f \circ d) (x,y) = 0 \Leftrightarrow  f(d(x,y)) = 0 \Leftrightarrow d(x,y) = 0 \Leftrightarrow x = y
\end{equation*}
\item Sean $x,y \in X$. Como $d$ es métrica en $X$, note lo siguiente
\begin{equation*}
    (f \circ d) (x,y) =  f(d(x,y)) =  f(d(y,x)) = (f \circ d) (y,x) 
\end{equation*}
\item Sean $x,y,x \in X$. Notemos que, como $f$ es estrictamente creciente, y además subaditiva s.t.q
\begin{equation*}
    (f \circ d) (x,z) =  f(d(x,z)) \leqslant f(d(x,y)+d(y,z)) \leqslant f(d(x,y)) + f(d(y,z))
\end{equation*}
\begin{equation*}
    = (f \circ d) (x,y) + (f \circ d) (y,z) \Rightarrow (f \circ d) (x,z) \leqslant  (f \circ d) (x,y) + (f \circ d) (y,z) 
\end{equation*}
\end{enumerate} 
$\therefore f \circ d$ es una métrica en $X$
\end{proof}

\item Sea $X$ un conjunto no vacío y $d : X \times X \to  [0, \infty] $ una función que cumple las siguientes condiciones $\: \forall \: x, y , z \in X$
\begin{enumerate}
    \item $d(x,y)=0 \Leftrightarrow x = y$
    \item $d(x,y) = d(y,x)$
    \item $d(x, z) \geqslant d(x, y) + d(y, z)$
\end{enumerate}

Pruebe que bajo estas condiciones X tiene  únicamente un punto

\begin{proof}
    Supongamos $x, y \in X \backepsilon x \neq y$
    
    Por la propiedad (a) s.t.q. $d(x,x)=0$

    Por la desigualdad triangular de la hipótesis s.t.q.
    \begin{equation*}
        d(x, x) \geqslant d(x, y) + d(y, x) =  0 \geqslant 2 \cdot d(y, x) = 0 \geqslant d(x,y)
    \end{equation*}
    
    Como $d : X \times X \to  [0, \infty] \Rightarrow  d(x,y) = 0$. Pero por (a) $d(x,y) = 0 \Leftrightarrow x = y \Rightarrow \Leftarrow$ 

    Pero esto es una contradicción, ya que supusimos a $x \neq y  \therefore X$ tiene únicamente un punto
\end{proof}

\item En el conjunto $\Z^+$ de los enteros positivos tomemos:

\begin{equation*}
        \rho(n,m) = 
     \begin{cases}
              0 & \text{si } m = n\\
              1+\frac{1}{n+m} & \text{si } m \neq n
     \end{cases}
    \end{equation*}

Demuestre que $\rho$ es una métrica.

\begin{proof}
\begin{enumerate}[label=(\subscript{D}{{\arabic*}})]
\item Sea $n, m \in \Z^+$ Si $m = n \Rightarrow \rho(n,m) \geqslant 0$. Si $m \neq n \Rightarrow \rho(n,m) \geqslant 0$, ya que 
\begin{equation*}
    \rho(n,m) = 1+\frac{1}{n+m} \geqslant 0
\end{equation*}
\item Se sigue de la definición de $\rho(n,m)$
\item Sea $n, m \in \Z^+$
\begin{equation*}
        \rho(n,m) = 
     \begin{cases}
              0 & \text{si } m = n\\
              1+\frac{1}{n+m} & \text{si } m \neq n
     \end{cases} = \begin{cases}
              0 & \text{si } n = m\\
              1+\frac{1}{m+n} & \text{si } n \neq m 
     \end{cases} 
     = \rho(m,n)
    \end{equation*}
\item Sea $n, m, c \in \Z^+$. Veamos que $\rho(n,c) \leqslant \rho(n,m) + \rho(m,c)$. Supongamos que $n \neq c$, ya que ese caso es trivial.
\begin{equation*}
    \rho(n,c) =  1+\frac{1}{n+c} \leqslant 2+\frac{1}{n+m}+\frac{1}{m+c} 
\end{equation*}
Esto porque $n, m , c \in \Z^+$
\begin{equation*}
    = \left(1+\frac{1}{n+m}\right)+ \left(1+\frac{1}{m+c}\right) = \rho(n,m) + \rho(m,c)
\end{equation*}
\end{enumerate}
$\therefore \rho$ es una métrica en $\Z^+$
\end{proof}

\item Sea $M_{n \times n}$ el espacio de matrices reales de tamaño $n \times n$. Demuestre que este conjunto es un espacio métrico con la función

\begin{equation*}
    \rho (A, B) = \sum_{j=1}^{n} \left(\sum_{i=1}^{n} \abs{{a}_{ij}-{b}_{ij}}\right)
\end{equation*}

Donde $A={a}_{ij}$ y $B={b}_{ij}$

\begin{proof}
\begin{enumerate}[label=(\subscript{D}{{\arabic*}})]
\item Sea $A, B \in M_{n \times n}$. Por propiedades de $\abs{\cdot}$ s.t.q.
\begin{equation*}
    \sum_{i=1}^{n} \abs{{a}_{ij}-{b}_{ij}} \geqslant 0 \Rightarrow \sum_{j=1}^{n} \left(\sum_{i=1}^{n} \abs{{a}_{ij}-{b}_{ij}}\right) \geqslant 0 \Rightarrow \rho (A, B) \geqslant 0 
\end{equation*}
\item Sea $A, B \in M_{n \times n}$
\begin{equation*}
    \rho (A, B) = 0 \Leftrightarrow \sum_{j=1}^{n} \left(\sum_{i=1}^{n} \abs{{a}_{ij}-{b}_{ij}}\right) = 0 \Leftrightarrow \sum_{i=1}^{n} \abs{{a}_{ij}-{b}_{ij}} 0 \Leftrightarrow \abs{{a}_{ij}-{b}_{ij}} = 0 \Leftrightarrow {a}_{ij} = {b}_{ij}
\end{equation*}
\item  Sea $A, B \in M_{n \times n}$
\begin{equation*}
    \rho (A, B) = \sum_{j=1}^{n} \left(\sum_{i=1}^{n} \abs{{a}_{ij}-{b}_{ij}}\right) = \sum_{j=1}^{n} \left(\sum_{i=1}^{n} \abs{{b}_{ij}-{a}_{ij}}\right) = \rho(B,A)
\end{equation*}
\item Sea $A, B, C \in M_{n \times n}$
\begin{equation*}
     \rho (A, B) = \sum_{j=1}^{n} \left(\sum_{i=1}^{n} \abs{{a}_{ij}-{b}_{ij}}\right) = \sum_{j=1}^{n} \left(\sum_{i=1}^{n} \abs{{a}_{ij}-{c}_{ij}+{c}_{ij}-{b}_{ij}}\right)
\end{equation*}
\begin{equation*}
    \leqslant \sum_{j=1}^{n} \left(\sum_{i=1}^{n} \left( \abs{{a}_{ij}-{c}_{ij}}+\abs{{c}_{ij}-{b}_{ij}} \right)\right) = \sum_{j=1}^{n} \left(\sum_{i=1}^{n} \abs{{a}_{ij}-{c}_{ij}}+\sum_{i=1}^{n} \abs{{c}_{ij}-{b}_{ij}}\right)
\end{equation*}
\begin{equation*}
    = \sum_{j=1}^{n} \left(\sum_{i=1}^{n} \abs{{a}_{ij}-{c}_{ij}}\right)+\sum_{j=1}^{n} \left(\sum_{i=1}^{n} \abs{{c}_{ij}-{b}_{ij}}\right) = \rho(A,C) + \rho(C,B)
\end{equation*}
\end{enumerate}
$\therefore \rho (A, B)$ es una métrica en $M_{n \times n}$
\end{proof}

\item Demuestra que  ${\norm{x}}_{\infty} = \max \{ \abs{{x}_{i}} \mid i = 1,...,n\}$ y ${\norm{x}}_{1} = \sum\limits_{i=1}^{n} \abs{{x}_{i}}$ donde $x=(x_1,...,x_n) \in \R^n$ son normas en $\R^n$

\begin{proof}
\begin{enumerate}[label=(\subscript{N}{{\arabic*}})]
\item $\forall \: (x_1,...,x_n) \in \R^n$ arbitrario s.t.q. $\abs{{x}_{i}} \geqslant 0 \: \forall \: i = 1,...,n$

\begin{equation*}
    \Rightarrow  \max \{ \abs{{x}_{i}} \mid i = 1,...,n\} ={\norm{x}}_{\infty} \geqslant 0
\end{equation*}

\item 

\begin{equation*}
    {\norm{x}}_{\infty} = 0 = \max \{ \abs{{x}_{i}} \mid i = 1,...,n\} \Leftrightarrow \abs{{x}_{i}} = 0 \: \forall \: i = 1,...,n \Leftrightarrow {x}_{i} = (0,...,0)
\end{equation*}

\item Sea $(x_1,...,x_n) \in \R^n$ y $\lambda \in \R$ arbitrarios

\begin{equation*}
    \Rightarrow {\norm{\lambda x}}_{\infty}  = \max \{ \abs{{\lambda x}_{i}} \mid i = 1,...,n\} = \max \{ \abs{\lambda } \abs{{\lambda x}_{i}} \mid i = 1,...,n\}
\end{equation*}

\begin{lema}
    Probaremos que, en general, $\forall \: \varepsilon > 0 $ y $y_1,...,y_n \in \R^n$ s.t.q. $\max \{ \varepsilon \cdot y_i \mid i = 1,...,n\} = \varepsilon \cdot \max \{  y_i \mid i = 1,...,n\}$ 
\end{lema}

\begin{proof}
    Supongamos que $\varepsilon \cdot y_j = \max \{ \varepsilon \cdot y_i \mid i = 1,...,n\}$ para algún índice $j \in \{ 1,...,n \} \Rightarrow \: \forall \: i = 1,...,n $ s.t.q.

    \begin{equation*}
        y_i \leqslant \varepsilon \cdot y_i \leqslant \varepsilon \cdot y_j
    \end{equation*}

    Como $\varepsilon > 0 \Rightarrow y_i \leqslant y_j \: \forall \:   i = 1,...,n$ , es decir que

    \begin{equation*}
        y_j = \max \{ y_i \mid i = 1,...,n\}
    \end{equation*}

    Así que, como $ y_i \leqslant y_j \leqslant \varepsilon \cdot y_i \leqslant \varepsilon \cdot y_j$ , s.t.q.

    \begin{equation*}
        \varepsilon \cdot \max \{ y_i \mid i = 1,...,n\} \leqslant \max \{ \varepsilon \cdot y_i \mid i = 1,...,n\}
    \end{equation*}

    Por otra parte, como $y_j = \max \{ y_i \mid i = 1,...,n\} \Rightarrow \: \forall \: i = 1,...,n $ s.t.q.
    
    \begin{equation*}
         \varepsilon \cdot y_i \leqslant \varepsilon \cdot y_j
    \end{equation*}

    Como $\varepsilon > 0 \Rightarrow$

    \begin{equation*}
        \max \{ \varepsilon \cdot y_i \mid i = 1,...,n\} \leqslant \varepsilon \cdot y_j 
    \end{equation*}
    \begin{equation*}
        \max \{ \varepsilon \cdot y_i \mid i = 1,...,n\} \leqslant \varepsilon \cdot \max \{ y_i \mid i = 1,...,n\} 
    \end{equation*}
    \begin{equation*}
        \Rightarrow max \{ \varepsilon \cdot y_i \mid i = 1,...,n\} = \varepsilon \cdot \max \{ y_i \mid i = 1,...,n\} 
    \end{equation*}
\end{proof}

Para $\lambda \in \R$ tenemos los siguientes casos

\begin{enumerate}
    \item $\lambda = 0$

    En este caso $\lambda \cdot x = \vec{0}$ y por ($N_2$) s.t.q.

    \begin{equation*}
        {\norm{\lambda x}}_{\infty} = 0 = 0 \cdot {\norm{ x}}_{\infty}
    \end{equation*}

    \item $\lambda \neq 0 \Rightarrow \abs{\lambda} > 0$

    Por el \textbf{Lema} anterior s.t.q.

    \begin{equation*}
        \max \{ \abs{\lambda } \abs{{\lambda x}_{i}} \mid i = 1,...,n\} = \abs{\lambda } \max \{  \abs{{\lambda x}_{i}} \mid i = 1,...,n\} = \abs{\lambda } {\norm{ x}}_{\infty}
    \end{equation*}
    
\end{enumerate}

\item Sean $x=(x_1,...,x_n)$ y $y=(y_1,...,y_n) \in \R^n$ arbitrarios. Por definición de ${\norm{ \cdot}}_{\infty}$ s.t.q.

\begin{equation*}
    {\norm{ x+y}}_{\infty} = \max \{ \abs{{x}_{i}+{y}_{i}} \mid i = 1,...,n\}
\end{equation*}

Supongamos que $\abs{x_j + y_j} = \max \{ \abs{{x}_{i}+{y}_{i}} \mid i = 1,...,n\}$

Como $\abs{x_j + y_j} \leqslant \abs{x_j} + \abs{y_j}$

\begin{equation*}
    \max \{ \abs{{x}_{i}+{y}_{i}} \mid i = 1,...,n\} \leqslant \abs{x_j} + \abs{y_j}
\end{equation*}
\begin{align*}
        \abs{x_j} \leqslant \max \{ \abs{{x}_{i}} \mid i = 1,...,n\} & & \text{y} & & \abs{y_j} \leqslant \max \{ \abs{{y}_{i}} \mid i = 1,...,n\}
\end{align*}

Por lo cual, por transitividad de $\leqslant$

\begin{equation*}
    \max \{ \abs{{x}_{i}+{y}_{i}} \mid i = 1,...,n\} \leqslant \max \{ \abs{{x}_{i}} \mid i = 1,...,n\} + \max \{ \abs{{y}_{i}} \mid i = 1,...,n\} 
\end{equation*}
\begin{equation*}
    \Rightarrow {\norm{ x+y}}_{\infty} \leqslant  {\norm{x}}_{\infty} +  {\norm{y}}_{\infty}
\end{equation*}
\end{enumerate}    
$\therefore {\norm{x}}_{\infty} = \max \{ \abs{{x}_{i}} \mid i = 1,...,n\}$ es una norma en $\R^n$
\end{proof}

\begin{proof}
\begin{enumerate}[label=(\subscript{N}{{\arabic*}})]
\item $\forall \: (x_1,...,x_n) \in \R^n$ arbitrario s.t.q. $\abs{{x}_{i}} \geqslant 0 \: \forall \: i = 1,...,n$
\begin{equation*}
    {\norm{x}}_{1} = \sum\limits_{i=1}^{n} \abs{{x}_{i}}  \geqslant 0 
\end{equation*}
\item 
\begin{equation*}
    {\norm{x}}_{1} = 0 \Leftrightarrow  \sum\limits_{i=1}^{n} \abs{{x}_{i}} = 0 \Leftrightarrow {x}_{i} = 0 \: \forall \: i = 1,...n
\end{equation*}
\item Sea $x= (x_1,...,x_n) \in \R^n$ y $\lambda \in \R$
\begin{equation*}
    {\norm{\lambda x}}_{1} = \sum\limits_{i=1}^{n} \abs{\lambda {x}_{i}} = \sum\limits_{i=1}^{n} \abs{\lambda} \abs{{x}_{i}} = \abs{\lambda} \sum\limits_{i=1}^{n}  \abs{{x}_{i}} = \abs{\lambda}{\norm{ x}}_{1}
\end{equation*}
\item  Sea $x= (x_1,...,x_n), y= (y_1,...,y_n) \in \R^n$
\begin{equation*}
    {\norm{x+y}}_{1} = \sum\limits_{i=1}^{n}  \abs{{x}_{i}+{y}_{i}} \leqslant \sum\limits_{i=1}^{n}  \left( \abs{{x}_{i}} + \abs{{y}_{i}} \right) = \sum\limits_{i=1}^{n}  \abs{{x}_{i}} + \sum\limits_{i=1}^{n}  \abs{{y}_{i}} = {\norm{x}}_{1} + {\norm{y}}_{1}
\end{equation*}
\end{enumerate}
$\therefore {\norm{x}}_{1}$ es una norma en $\R^n$
\end{proof}

\item Sea $\sum_{n=1}^{\infty} a_n$ an una serie convergente de números reales positivos. En el conjunto $E$ de todas las sucesiones de números reales $x ={(x_n)}_{n \in \N} $ definimos

\begin{equation*}
    d(x,y) = \sum_{n=1}^{\infty} a_n \frac{1+ \abs{x_n-y_n}}{\abs{x_n-y_n}}
\end{equation*}

\begin{enumerate}
    \item Demostrar que $d$ es una métrica en $E$
    \item ¿Se puede introducir una norma en el espacio $E$ de tal modo que se cumpla la igualdad $d(x,y) = \norm{x-y}$?
    \item Dar un ejemplo de una sucesión $({x}_{n}^{1},{x}_{n}^{2},...){x}_{n}^{i} \in \R$ que converja en el espacio $E$, que pertenezca al espacio $\ell_2$ pero que no converja en el espacio $\ell_2$.
\end{enumerate}

\begin{proof}
    Primero veamos que $d : E \times E \to \R$ está bien definida, es decr, que $\forall \: x = {(x_n)}_{n \in \N} $, $  y = {(y_n)}_{n \in \N} \in E$ la serie 
    \begin{equation*}
        \sum_{n=1}^{\infty} a_n \frac{1+ \abs{x_n-y_n}}{\abs{x_n-y_n}}
    \end{equation*}

    converge en $\R$

    $\Rightarrow$ supongamos que $x, y \in E$ arbitrarios. Notemos que 
    \begin{equation*}
        0 \leqslant \frac{1+ \abs{x_n-y_n}}{\abs{x_n-y_n}} \leqslant 1 \: \: \: \: \: \: \: \: \: \: \: \: \forall \: n \in \N
    \end{equation*}
    Así que 
    \begin{equation*}
        0 \leqslant a_n \frac{1+ \abs{x_n-y_n}}{\abs{x_n-y_n}} \leqslant a_n \: \: \: \: \: \: \: \: \: \: \: \: \forall \: n \in \N
    \end{equation*}
    \begin{equation*}
        0 \leqslant  \sum_{n=1}^{\infty} a_n \frac{1+ \abs{x_n-y_n}}{\abs{x_n-y_n}} \leqslant  \sum_{n=1}^{\infty} a_n \: \: \: \: \: \: \: \: \: \: \: \: \forall \: n \in \N
    \end{equation*}
    $\Rightarrow$ por criterios de convergencia $\sum\limits_{n=1}^{\infty} a_n \frac{1+ \abs{x_n-y_n}}{\abs{x_n-y_n}}$ converge a un número en $\R$
\begin{enumerate}[label=(\subscript{D}{{\arabic*}})]
\item En la demostración de que $d$ está bien definida se hace evidente
\item Supongamos que $x ={(x_n)}_{n \in \N}, y ={(y_n)}_{n \in \N} \in E$
\begin{equation*}
    d(x,y) = \sum_{n=1}^{\infty} a_n \frac{1+ \abs{x_n-y_n}}{\abs{x_n-y_n}} = 0 \Leftrightarrow \: \forall \: n \in \N \Rightarrow a_n \frac{1+ \abs{x_n-y_n}}{\abs{x_n-y_n}} = 0
\end{equation*}
\begin{equation*}
    \Leftrightarrow \abs{x_n-y_n} = 0 \Leftrightarrow x_n = y_n \: \forall \: n \in \N \Rightarrow d(x,y) = 0 \Leftrightarrow x = y
\end{equation*}
\item Sean  $x ={(x_n)}_{n \in \N}, y ={(y_n)}_{n \in \N} \in E$
\begin{equation*}
    d(x,y) = \sum_{n=1}^{\infty} a_n \frac{1+ \abs{x_n-y_n}}{\abs{x_n-y_n}} =  \sum_{n=1}^{\infty} a_n \frac{1+ \abs{y_n-x_n}}{\abs{y_n-x_n}} = d(y,x)
\end{equation*}
\item Sean  $x ={(x_n)}_{n \in \N}, y ={(y_n)}_{n \in \N}, z ={(z_n)}_{n \in \N} \in E$

En el ejercicio \textbf{1.} se demostro que $\forall \: n \in \N$ s.t.q.
\begin{equation*}
    \frac{1+ \abs{x_n-y_n}}{\abs{x_n-y_n}} \leqslant \frac{1+ \abs{x_n-z_n}}{\abs{x_n-z_n}} + \frac{1+ \abs{z_n-y_n}}{\abs{z_n-y_n}}
\end{equation*}
Así que
\begin{equation*}
    d(x,y) = \sum_{n=1}^{\infty} a_n \frac{1+ \abs{x_n-y_n}}{\abs{x_n-y_n}}  \leqslant \sum_{n=1}^{\infty} a_n \left( \frac{1+ \abs{x_n-z_n}}{\abs{x_n-z_n}} + \frac{1+ \abs{z_n-y_n}}{\abs{z_n-y_n}} \right)
\end{equation*}
\begin{equation*}
    =\sum_{n=1}^{\infty} \left(  a_n \frac{1+ \abs{x_n-z_n}}{\abs{x_n-z_n}} +  a_n \frac{1+ \abs{z_n-y_n}}{\abs{z_n-y_n}} \right) 
\end{equation*}
\begin{equation*}
      = \sum_{n=1}^{\infty}  a_n \frac{1+ \abs{x_n-z_n}}{\abs{x_n-z_n}} + \sum_{n=1}^{\infty} a_n \frac{1+ \abs{z_n-y_n}}{\abs{z_n-y_n}} = d(x,z) + d(z,y)
\end{equation*}
\end{enumerate}
$\therefore d$ es una métrica en $E$
\end{proof}

\begin{proof}
    Para que $d(x,y) = \norm{x-y}$ sea compatible se debe de cumplir $d(\lambda x, \lambda y ) = \abs{\lambda} d(x,y)$

    \begin{equation*}
        d(\lambda x, \lambda y ) = \sum_{n=1}^{\infty} a_n \frac{1+ \abs{\lambda x_n-\lambda y_n}}{\abs{\lambda x_n- \lambda y_n}} = \sum_{n=1}^{\infty} a_n \frac{1+ \abs{\lambda \cdot (x_n-y_n)}}{\abs{\lambda \cdot (x_n-y_n)}}
    \end{equation*}
    \begin{equation*}
        \sum_{n=1}^{\infty} a_n \frac{1+ \abs{\lambda} \abs{ x_n-y_n}}{\abs{\lambda} \abs{ x_n-y_n}} \neq \abs{\lambda} 
        \sum_{n=1}^{\infty} a_n \frac{1+  \abs{ x_n-y_n}}{\abs{ x_n-y_n}}
    \end{equation*}
\end{proof}

\item Sea $(E, \norm{\cdot})$ un espacio normado cuya norma procede de un producto interior.
\begin{enumerate}
    \item Demuestra que la igualdad $\norm{x+y} = \norm{x} + \norm{y}$ implica que los vectores $x$ y $y$ son linealmente dependientes.
    \item Comprueba la identidad del paralelogramo
    \item Infiera que la norma infinito y norma uno no provienen de un producto interior
\end{enumerate}

\begin{proof}
    Recordemos la desigualdad C-S. Sea ($V, \langle \: , \: \rangle$) un espacio con producto interior, $ \implies  \forall \: x,y \in V$ s.t.q.

    \begin{equation*}
        \abs{\langle x,y \rangle} \leqslant \sqrt{\langle x,x \rangle} \cdot \sqrt{\langle y,y \rangle}
    \end{equation*}

    Para probar el inciso a) es sufciente probar el siguiente lema

    \begin{lema}
        En las condiciones de C-S
        \begin{equation*}
            \abs{\langle x,y \rangle} = \sqrt{\langle x,x \rangle} \cdot \sqrt{\langle y,y \rangle} \Leftrightarrow x,y \text{ es linealmente dependiente}
        \end{equation*}
    \end{lema}
    \begin{proof}
        $\Rightarrow$ Supongamos que $ \abs{\langle x,y \rangle} = \sqrt{\langle x,x \rangle} \cdot \sqrt{\langle y,y \rangle}$

        Supongamos que $x \neq 0 \neq y$, ya que cuando son $x=0$ o $y=0$ trivialmente se cumple que son linealmente dependiente.

        \begin{equation*}
            \Rightarrow {\langle x,y \rangle}^{2} = \langle x,x \rangle \cdot \langle y,y \rangle
        \end{equation*}

        Por ello, s.t.q.
        \begin{equation*}
            \langle x,x \rangle = \frac{{\langle x,y \rangle}^{2}}{\langle y,y \rangle} \Rightarrow \langle x,x \rangle - \frac{{\langle x,y \rangle}^{2}}{\langle y,y \rangle} = 0 \Rightarrow \langle x,x \rangle - \frac{\langle x,y \rangle }{\langle y,y \rangle } \cdot \langle x,y \rangle 
        \end{equation*}
        \begin{equation*}
            =  \langle x,x \rangle - 2 \cdot \frac{\langle x,y \rangle }{\langle y,y \rangle } \cdot \langle x,y \rangle + \frac{\langle x,y \rangle }{\langle y,y \rangle } \cdot \langle x,y \rangle 
        \end{equation*}
        \begin{equation*}
            = \langle x,x \rangle - 2 \cdot \frac{\langle x,y \rangle }{\langle y,y \rangle } \cdot \langle x,y \rangle + \frac{{\langle x,y \rangle}^{2} }{\langle y,y \rangle } \frac{\langle y,y \rangle}{\langle y,y \rangle} = \langle x,x \rangle - 2 \cdot \frac{\langle x,y \rangle }{\langle y,y \rangle } \cdot \langle x,y \rangle + \frac{{\langle x,y \rangle}^{2} }{{\langle y,y \rangle}^{2}} \langle y,y \rangle
        \end{equation*}

        Definimos a $\lambda_0 = \frac{\langle x,y \rangle }{\langle y,y \rangle } \in \R \Rightarrow$
        \begin{equation*}
            0 = \langle x,x \rangle - 2\lambda_0 \langle x,y \rangle + {\lambda}_{0}^{2} \langle y,y \rangle
        \end{equation*}
        \begin{equation*}
            = \langle x - {\lambda}_{0}y, x - {\lambda}_{0}y \rangle = 0 \Leftrightarrow x - {\lambda}_{0}y = 0 \Rightarrow x = {\lambda}_{0}y  \Leftrightarrow \text{ son l.d.}
        \end{equation*}

        $\Leftarrow$ Supongamos que son l.d. $\Rightarrow \: \exists \: \alpha, \beta \in \R \backepsilon$ no sn ambos cero y $\alpha x + \beta y = 0$. Supongamos que $\alpha \neq 0 \Rightarrow x = \frac{ - \beta y }{\alpha}$

        Luego

        \begin{equation*}
            \abs{\langle x,y \rangle} = \abs{\langle \frac{ - \beta y }{\alpha}, y \rangle } = \abs{\frac{ - \beta  }{\alpha} \langle y , y \rangle} = \abs{\frac{ - \beta  }{\alpha}} \abs{\langle y, y \rangle}
        \end{equation*}
        Por otra parte 
        \begin{equation*}
            \sqrt{\langle x,x \rangle} = \sqrt{\langle  \frac{ - \beta y }{\alpha},  \frac{ - \beta y }{\alpha} \rangle} = \sqrt{{\left( \frac{ - \beta }{\alpha} \right)}^{2}  \langle y , y \rangle } = \abs{\frac{ - \beta  }{\alpha}} \sqrt{\langle y, y \rangle}
        \end{equation*}
        \begin{equation*}
            \Rightarrow \sqrt{\langle x,x \rangle} \cdot \sqrt{\langle y, y \rangle} = \abs{\frac{ - \beta  }{\alpha}} {(\sqrt{\langle y, y \rangle})}^{2} = \abs{\frac{ - \beta  }{\alpha}} \abs{\langle y, y \rangle}
        \end{equation*}
        $\Rightarrow \abs{\langle x,y \rangle} = \sqrt{\langle x,x \rangle} \cdot \sqrt{\langle y,y \rangle}$
    \end{proof}

    Supongamos que $\norm{x+y} = \norm{x} + \norm{y}$. Como $\norm{\cdot}$ viene de un producto interior $\Rightarrow$

    \begin{equation*}
        \sqrt{\langle x+y, x+y \rangle} = \sqrt{\langle x,x \rangle} + \sqrt{\langle y,y \rangle}
    \end{equation*}

    Luego 

    \begin{equation*}
        \Rightarrow \langle x+y, x+y \rangle = \langle x,x \rangle + 2\sqrt{\langle x,x \rangle \cdot \langle y,y \rangle} + \langle y,y \rangle
    \end{equation*}

    Vamos a expandir el producto interior del lado izquierdo de la ecaución

    \begin{equation*}
        \langle x+y, x+y \rangle = \langle x, x+y \rangle + \langle y, x+y \rangle = \langle x, x\rangle + \langle x, y\rangle + \langle y, x\rangle  + \langle y,y \rangle 
    \end{equation*}
    \begin{equation*}
        \Rightarrow \langle x, x\rangle + 2 \langle x, y\rangle + \langle y,y \rangle =  \langle x,x \rangle + 2\sqrt{\langle x,x \rangle \cdot \langle y,y \rangle} + \langle y,y \rangle
    \end{equation*}
    \begin{equation*}
        \Rightarrow 2 \langle x, y\rangle  = 2\sqrt{\langle x,x \rangle \cdot \langle y,y \rangle} \Rightarrow  \langle x, y\rangle  = \sqrt{\langle x,x \rangle \cdot \langle y,y \rangle} = \sqrt{\langle x,x \rangle } \sqrt{\langle y,y \rangle}
    \end{equation*}

    Pero notemos que $\sqrt{\langle x,x \rangle } \sqrt{\langle y,y \rangle} = \norm{x} \norm{y}$, que por ($N_1$)  $\geqslant 0 \Rightarrow \langle x, y\rangle = \abs{\langle x, y\rangle}$

    $\therefore x, y $ son linealmente dependientes por el Lema anterior.
\end{proof}

\begin{proof}
    Sea $x, y \in E$. Por definición de $E \Rightarrow$

    \begin{equation*}
       {\norm{x+y}}^{2} + {\norm{x-y}}^{2} =  {\sqrt{\langle x + y , x + y \rangle}}^{2}+ {\sqrt{\langle x - y , x - y \rangle}}^{2}
    \end{equation*}
    \begin{equation*}
        = \langle x + y , x + y \rangle + \langle x - y , x - y \rangle = \langle x , x + y \rangle + \langle y , x + y \rangle + \langle x  , x - y \rangle - \langle y , x - y \rangle
    \end{equation*}
    \begin{equation*}
        = \langle x , x  \rangle + \langle x , y \rangle+ \langle y , x \rangle + \langle y ,y \rangle + \langle x  , x \rangle - \langle x  , y \rangle - \langle y , x \rangle + \langle y , y \rangle
    \end{equation*}
    \begin{equation*}
        = 2 \langle x  , x \rangle + 2 \langle y, y \rangle = 2 {\norm{x}}^{2} + 2 {\norm{y}}^{2}
    \end{equation*}
    $\therefore$ se cumple la identidad del paralelogramo
\end{proof}

\begin{proof}
    \begin{enumerate}
        \item Notemos que la norma infinito no satisface la identidad del paralelogramo. Sea $x = (1,0,...,0) , y = (0,-1,0,...,0) \in \R^n$

        \begin{equation*}
            {\norm{x+y}}^{2} + {\norm{x-y}}^{2} = 2 {\norm{x}}^{2} + 2 {\norm{y}}^{2}
        \end{equation*}
        \begin{equation*}
            {\max \{ \abs{{x}_{i}+{y}_{i}} \mid i = 1,...,n\}}^{2} + {\max \{ \abs{{x}_{i}-{y}_{i}}\}}^{2} = 2 \cdot {\max \{ \abs{{x}_{i}}\}}^{2} +  2 \cdot {\max \{ \abs{{y}_{i}} \}}^{2}
        \end{equation*}
        \begin{equation*}
            1^2 + 1^2 = 2 \neq 2 \cdot 1^2 + 2 \cdot 1^2 = 4
        \end{equation*}
        \item Notemos que la norma uno no satisface la identidad del paralelogramo. Sea $x = (1,0,...,0)$,  $y = (0,1,0,...,0) \in \R^n$

        \begin{equation*}
        \sum_{i=1}^{n} \abs{x_i + y_i}^2 + \sum_{i=1}^{n} \abs{x_i - y_i}^2 = 2 \sum_{i=1}^{n} \abs{x_i}^2 + 2 \sum_{i=1}^{n} \abs{y_i}^2
    \end{equation*}
    \begin{equation*}
        2^2 + 2^2 = 8 \neq 2 \cdot 1^2 + 2 \cdot 1^2 = 4
    \end{equation*}
    \end{enumerate}
\end{proof}

\item Consideremos el polinomio $Q(x, y) = ax^2 + 2bxy + cy^2$ donde $a, b, c \in \R$, $a > 0$ y $ac - b^2 > 0$. Demuestra que la fórmula $\norm{(x, y)} = \sqrt{Q(x, y)} $ define en $\R^2$ una norma asociada a un producto interior.

\begin{proof}
    NO SE
\end{proof}

\item Sea $(X, d)$ un espacio métrico y $A \subseteq X$. La distancia de $A$ a $x \in X $ se define como ${d}^{\prime}(A,x) = \inf \{ d(y,x) \mid y \in A \}$. Demostrar que $\abs{{d}^{\prime}(A,x)-{d}^{\prime}(A,y)} \leqslant d(x,y)$

\begin{proof}
    Sea $x,y \in X \Rightarrow \forall \: z \in A$ s.t.q.

    \begin{equation*}
        {d}^{\prime}(A,x) \leqslant d(x,z) \leqslant d(x,y) + d(y,z)
    \end{equation*}
    \begin{equation*}
        {d}^{\prime}(A,y) \leqslant d(y,z) \leqslant d(y,x) + d(x,z)
    \end{equation*}

    Podemos tomar el ínfimo en $z$, ya que la desigualdad se cumple para todas las métricas

    \begin{equation*}
        {d}^{\prime}(A,x) \leqslant d(x,y) + {d}^{\prime}(A,y) \Rightarrow  {d}^{\prime}(A,x) - {d}^{\prime}(A,y)  \leqslant d(x,y) 
    \end{equation*}
    \begin{equation*}
        {d}^{\prime}(A,y)  \leqslant d(y,x) + {d}^{\prime}(A,x) \Rightarrow {d}^{\prime}(A,y) - {d}^{\prime}(A,x) \leqslant d(y,x)
    \end{equation*}

    Esto implica $\abs{{d}^{\prime}(A,x)-{d}^{\prime}(A,y)} \leqslant d(x,y)$
\end{proof}

\item Sea $(X,d)$ un espacio métrico y $\varnothing \neq Y \subseteq X$, se define al diámetro de $Y$ como $\text{diam} (Y) = \sup \{ d(y,w) \mid y, w \in Y\}$. Suponga que $(V, \norm{\cdot})$ es un espacio vectorial normado, que $d(x,y) = \norm{x-y}$ y que $A, B \subseteq V$ son no vacíos.¿Se cumplen las siguientes afirmaciones? argumente o dé un contraejemplo.

\begin{enumerate}
    \item Si $A \cap B \neq \varnothing \Rightarrow \text{diam} (A \cup B) \leqslant \text{diam} (A) + \text{diam} (B)$
    \item $\text{diam} (A \setminus B) = \text{diam} (A) - \text{diam} (B)$
    \item Si $A \cap B = \varnothing \Rightarrow \text{diam} (A \cup B) = \text{diam} (A) + \text{diam} (B)$
    \item Dado $u \in V$ definimos $A + u = \{a + u \mid a \in A \} \Rightarrow \text{diam} (A + u) = \text{diam} (A)$
    \item Si $\lambda \in \R$ definimos $\lambda A = \{ \lambda a \mid a \in A \} \Rightarrow \text{diam} (\lambda A) = \abs{\lambda} \text{diam} (A)$
    \item Si $A \subseteq B \Rightarrow \text{diam} (A) \leqslant \text{diam} (B)$
\end{enumerate}

\begin{proof}
    \begin{enumerate}
        \item $\forall \: x,y \in A \cup B$ se tienen los siguientes casos
        \begin{enumerate}
            \item Si $x , y \in A \Rightarrow d(x,y) \leqslant  \text{diam} (A) \leqslant  \text{diam} (A) +  \text{diam} (B)$
            \item Si $x , y \in B \Rightarrow d(x,y) \leqslant  \text{diam} (B) \leqslant  \text{diam} (B) +  \text{diam} (A)$
            \item SPG supongamos que $x \in A$ y que $y \in B \Rightarrow \: \exists \: z \in A \cap B $
            \begin{equation*}
                \Rightarrow d(x,y) \leqslant d(x,z) + d(z,y) \leqslant \text{diam} (A) +  \text{diam} (B)
            \end{equation*}
        \end{enumerate}
        \item Creo que no 
        \item Creo que no
        \item Creo que no
        \item Esto es cierto ya que en la definición de distancia se puede ver que proviene de una norma, y esto sucede si y solo si saca escalares en valor absoluto.
        
    \end{enumerate}
\end{proof}

\item Demuestra la Desigualdad de Hölder para series

\begin{proof}
    Sean $p, q \in [1, \infty] \backepsilon \frac{1}{p}+\frac{1}{q}= 1$

    Sea ${({x}_{m})}_{m=1}^{\infty} \in {\ell}_{p} (\R)$ y ${({y}_{m})}_{m=1}^{\infty} \in {\ell}_{q} (\R)$

    Aplicando la desigualdad de Hölder en $\R^n$ tenemos que $\forall \: n \in \N $

    \begin{equation*}
        \sum_{i=1}^{n} \abs{{x}_{i}{y}_{i}} \leqslant {\left(  \sum_{i=1}^{n} {\abs{{x}_{i}} }^{p} \right)}^{\frac{1}{p}} {\left(  \sum_{i=1}^{n} {\abs{{y}_{i}} }^{q} \right)}^{\frac{1}{q}} \leqslant {\left(  \sum_{i=1}^{\infty} {\abs{{x}_{i}} }^{p} \right)}^{\frac{1}{p}} {\left(  \sum_{i=1}^{\infty} {\abs{{y}_{i}} }^{q} \right)}^{\frac{1}{q}} = M 
    \end{equation*}
    Notemos que $M$ es un número real, porque pertenece a $ {\ell}_{p} (\R)$, por lo tanto converge
    \begin{equation*}
        \Rightarrow \forall \: n \in \N \Rightarrow  \sum_{i=1}^{n} \abs{{x}_{i}{y}_{i}} \leqslant M \Rightarrow \sum_{n=1}^{\infty} \abs{{x}_{n}{y}_{n}} \text{ converge }
    \end{equation*}
    Si se tiene una sucesión de sumas parciales acotadas y mayores iguales a cero, y se encuentra una cota superior, entonces la serie converge
    \begin{equation*}
        \Rightarrow \sum_{n=1}^{\infty} \abs{{x}_{n}{y}_{n}} \leqslant {\left(  \sum_{n=1}^{\infty} {\abs{{x}_{n}} }^{p} \right)}^{\frac{1}{p}} {\left(  \sum_{n=1}^{\infty} {\abs{{y}_{n}} }^{q} \right)}^{\frac{1}{q}}
    \end{equation*}
    Pero esta es la desigualdad de Hölder para series 
\end{proof}
\end{enumerate}
\end{document}